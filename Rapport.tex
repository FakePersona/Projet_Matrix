\documentclass[a4paper,10pt]{article}
\usepackage[utf8]{inputenc}

%opening
\title{PROG2: Inversion de matrices}
\author{Rémi Hutin \and Rémy Sun}
\date{26 février 2016}


\begin{document}

\maketitle

\begin{abstract}

\end{abstract}

\section{Implémentation de matrices}

%% Blah blah...

\subsection{Extraction de sous-matrice}

Il est nécessaire d'extraire une sous-matrice en retirant les lignes d'indice $a$ et les colonnes d'indice $b$. 

Ceci est fait en créant une matrice carrée de taille $n-1$, qu'on remplit par parcours de cette matrice en tirant parti du fait que l'expression
($i$ >= a) renvoie 1 si $i \leq a$ ce qui permet d'engendrer un décalage de ligne/colonne quand nécessaire.

\section{Optimisation: implémentation des sous-matrices}

\end{document}
